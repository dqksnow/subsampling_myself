\documentclass[letterpaper,12pt,hidelinks]{article}
\usepackage{configuration, natbib,mathtools,rotating, showlabels}
\usepackage[T1]{fontenc}
\graphicspath{{figures/}}
\usepackage{cases}



\title{R package}
\author{}
\date{}

\begin{document}
\maketitle

\section{Logistic regression}

\subsection{Sampling with replacement}

The optimal subsampling probabilities are
\begin{align*}
    \pi_i^\mmse &= \frac{|y_i - p_i(\hbeta_\mle)|\|\M_N^{-1}\x_i\|}{\sumN |y_i - p_i(\hbeta_\mle)|\|\M_N^{-1}\x_i\|}, \\
    \pi_i^\mvc &= \frac{|y_i - p_i(\hbeta_\mle)|\|\x_i\|}{\sumN |y_i - p_i(\hbeta_\mle)|\|\x_i\|}\\
\end{align*}
where
\begin{align*}
    p_i(\bbeta) &= \frac{\exp(\x_i\tp \bbeta)}{1 + \exp(\x_i\tp \bbeta)}, \\
    \M_N &= \oneN \sumN p_i(\hbeta_\mle)\{1 - p_i(\hbeta_\mle)\} \x_i\x_i\tp.
\end{align*}
To accelerate the computation, $\M_N$ is calculated by the pilot sample.

\subsection{Poisson subsampling}

The optimal subsampling probabilities for Poisson subsampling is approximated by
\begin{align*}
    \pi_i^\mmse &= \frac{\{|y_i - p_i(\hbeta_\mle)|\|{\M_N^{*0}}^{-1}\x_i\|\}\wedge H^{*\mmse}}{N\Phi^{*\mmse}} ,\\
    \pi_i^\mvc &= \frac{\{|y_i - p_i(\hbeta_\mle)|\|\x_i\|\} \wedge H^{*\mvc}}{N\Phi^{*\mvc}}
\end{align*}
where
\begin{align*}
    H^{*\mmse} &= \{|y_i^{*0} - p_i^{*0}(\hbeta_\mle)|\|{\M_N^{*0}}^{-1}\x_i^{*0}\|\}_{\frac{r}{bN}}, \\
    H^{*\mvc} &= \{|y_i^{*0} - p_i^{*0}(\hbeta_\mle)|\|\x_i^{*0}\|\}_{\frac{r}{bN}}, \\
    \Phi^{*\mmse} &= \sum_{i=1}^{r_0^*} \frac{\{|y_i^{*0} - p_i^{*0}(\hbeta_\mle)|\|{\M_N^{*0}}^{-1}\x_i^{*0}\|\} \wedge H^{*\mmse}}{N r_0^* \pi_i^{*0}}, \\
    \Phi^{*\mvc} &= \sum_{i=1}^{r_0^*} \frac{\{|y_i^{*0} - p_i^{*0}(\hbeta_\mle)|\|\x_i^{*0}\|\} \wedge H^{*\mvc}}{N r_0^* \pi_i^{*0}},
\end{align*}
and the subscript $\frac{r}{bN}$ means the $1 - \frac{r}{bN}$ quantile of the vector.

\subsection{Unweighted estimator}

To correct the bias, the unweighted estimator is 
\begin{align*}
    \hbeta_{\mathrm{UW}} = \hbeta_{\mathrm{LLK}} + \hbeta_{\mathrm{PLT}}.
\end{align*}

For the pilot sample, 
\begin{align*}
    \hbeta_{\mathrm{UW}} = \hbeta_{\mathrm{LLK}} + \{\log(n_1/n_0), 0, ..., 0\}\tp.
\end{align*}

The final estimator is obtained by combining the subsample estimators from two 
stages.
\begin{align*}
    \hbeta_{\mathrm{CMB}} = &\left\{\sum_{i=1}^{r_0} \phi_i^{*0}(\hbeta_{\mathrm{LLK}}^{*0}) \x_i^{*0} {\x_i^{*0}}\tp + \sum_{i=1}^{r_1} \phi_i^{*1}(\hbeta_{\mathrm{LLK}}^{*1}) \x_i^{*1} {\x_i^{*1}}\tp\right\}^{-1}\\
    &\quad \left[ \left\{\sum_{i=1}^{r_0} \phi_i^{*0}(\hbeta_{\mathrm{LLK}}^{*0}) \x_i^{*0} {\x_i^{*0}}\tp\right\}\hbeta_{\mathrm{UW}}^{*0} + \left\{\sum_{i=1}^{r_1} \phi_i^{*1}(\hbeta_{\mathrm{LLK}}^{*1}) \x_i^{*1} {\x_i^{*1}}\tp\right\}\hbeta_{\mathrm{UW}}^{*1} \right]
\end{align*}
where $\phi_i = p(\x_i, \bbeta)\{ 1 - p(\x_i, \bbeta)\}$.


\subsection{Variance Estimation}

For weighted estimator, the variance-covariance matrix of the final estimator can be estiamted by
\begin{align*}
    \V = &\left\{\frac{1}{N}\sum_{i=1}^{r_0} \frac{\phi_i^{*0}(\hbeta_{\mathrm{LLK}}^{*0}) \x_i^{*0} {\x_i^{*0}}\tp}{\pi_i^{*0}} + \frac{1}{N}\sum_{i=1}^{r_1} \frac{\phi_i^{*1}(\hbeta_{\mathrm{LLK}}^{*1}) \x_i^{*1} {\x_i^{*1}}\tp}{\pi_i^{*1}}\right\}^{-1}\\
    &\quad \left[\frac{1}{N^2} \sum_{i=1}^{r_0} \frac{\left\{\psi_i^{*0}(\hbeta_{\mathrm{LLK}}^{*0}) \right\}^2 \x_i^{*0} {\x_i^{*0}}\tp}{(\pi_i^{*0})^2} + \frac{1}{N^2}\sum_{i=1}^{r_1} \frac{\left\{\psi_i^{*1}(\hbeta_{\mathrm{LLK}}^{*1})\right\}^2 \x_i^{*1} {\x_i^{*1}}\tp}{(\pi_i^{*1})^2} \right]\\
    &\quad\left\{\frac{1}{N}\sum_{i=1}^{r_0} \frac{\phi_i^{*0}(\hbeta_{\mathrm{LLK}}^{*0}) \x_i^{*0} {\x_i^{*0}}\tp}{\pi_i^{*0}} + \frac{1}{N}\sum_{i=1}^{r_1} \frac{\phi_i^{*1}(\hbeta_{\mathrm{LLK}}^{*1}) \x_i^{*1} {\x_i^{*1}}\tp}{\pi_i^{*1}}\right\}^{-1}
\end{align*}
where $\psi_i = y_i - p(\x_i, \bbeta)$.


For unweighted estimator, the variance-covariance matrix of the final estimator can be estimated by
\begin{align*}
    \V = &\left\{\sum_{i=1}^{r_0} \phi_i^{*0}(\hbeta_{\mathrm{LLK}}^{*0}) \x_i^{*0} {\x_i^{*0}}\tp + \sum_{i=1}^{r_1} \phi_i^{*1}(\hbeta_{\mathrm{LLK}}^{*1}) \x_i^{*1} {\x_i^{*1}}\tp\right\}^{-1}\\
    &\quad \left[\sum_{i=1}^{r_0} \left\{ \psi_i^{*0}(\hbeta_{\mathrm{LLK}}^{*0}) \right\}^2 \x_i^{*0} {\x_i^{*0}}\tp + \sum_{i=1}^{r_1} \left\{ \psi_i^{*1}(\hbeta_{\mathrm{LLK}}^{*1})\right\}^2 \x_i^{*1} {\x_i^{*1}}\tp \right]\\
    &\quad\left\{\sum_{i=1}^{r_0} \phi_i^{*0}(\hbeta_{\mathrm{LLK}}^{*0}) \x_i^{*0} {\x_i^{*0}}\tp + \sum_{i=1}^{r_1} \phi_i^{*1}(\hbeta_{\mathrm{LLK}}^{*1}) \x_i^{*1} {\x_i^{*1}}\tp\right\}^{-1}
\end{align*}


\subsection{Rare logistic}

The unknown estimator is $\btheta = \{\alpha, \bbeta\tp\}\tp$.

\subsubsection{Algorithm for rare logistic}

\begin{itemize}
    \item Pilot sampling: Randomly draw $r_0$ observations with case control 
    probabilities. Denote the drawn sample as 
    $\{\x_i^{*0}, y_i^{*0}, \pi_i^{*0}\}_{i=1}^{r_0^*}$. Compuate the pilot 
    estimator 
    $\htheta^{*0} = \htheta_{\mathrm{LLK}} + c(\log\frac{N_1}{N_0}$, 0, ... 0).
    \item Compute approximated optimal subsampling probabilities for all 
    control which are 
    \begin{align*}
        \pi_i^{\mathrm{opt}} = \frac{p_i(\htheta^{*0})\|\tilde\M_f^{-1} \z_i\|}{\red N\tilde\w},
    \end{align*}
    where 
    \begin{align*}
        \tilde\M_f = \oneN\sumN \frac{\exp\{ \x_i\tp \hbeta^{*0}\} \z_i\z_i\tp}{\pi_i^{*0}},
    \end{align*}
    under A-optimality criterion and $\tilde\M_f = \I$ under L-optimality criterion, 
$\z_i = (1, \x_i\tp)\tp$,  
\begin{align*}
    \tilde\w = \frac{1}{r_0^* N} \sum_{i=1}^{r_0^{*}} \frac{p^{*0}_i(\htheta^{*0})\|\tilde\M_f^{-1} \z^{*0}_i\|}{\pi_i^{*0}}.
\end{align*}
\item Negative sampling. Sample observations with probabilities 
$r \pi_i^{\mathrm{opt}} \wedge 1$. The subsampled estimator is 
\begin{align*}
    \htheta^{\mathrm{NS}} = \arg\max_{\bbeta} \sumN \delta_i 
    \left[y_i \z_i\tp \btheta - \log\left\{ 1 + \exp(\z_i\tp \btheta - \log [r \pi_i]) \right\}\right].
\end{align*}
where $\delta_i$ is the indicator variable for $i$-th observation, and 
$\pi_i$ is the probability for $i$-th observation to be included into 
the subsample. When $y_i = 1$, $\pi_i = 1$; and 
when $y_i = 0$, $\pi_i = r\pi_i^{\mathrm{opt}} \wedge 1$.
\end{itemize}

\subsection{Problems}

\begin{itemize}
    \item Double check the code for (LCC, Poisson, weighted estimator) VS 
    (LCC, Poisson, unweighted estimator). The simulation result is not as expected.
\end{itemize}


\end{document}